\documentclass[xcolor={dvipsnames}]{beamer}

%%% PACKAGES %%%
\usepackage{graphicx}
\usepackage{tabularx}
\usepackage{booktabs}
\usepackage{multirow}
%\usepackage[usenames,dvipsnames]{color}
\usepackage{textpos}
\usepackage{hyperref}




%%% FONT %%%
\renewcommand*{\familydefault}{phv}


%%% BEAMER THEME %%%
\usetheme{default}
\usecolortheme{whale}
\usecolortheme{orchid}
\setbeamertemplate{navigation symbols}{} 
\setbeamertemplate{footline}[frame number]
\addtobeamertemplate{frametitle}{}{%
\begin{textblock*}{100mm}(.99\textwidth,-.9cm)
\fboxsep=1pt%padding thickness
\fboxrule=1pt%border thickness
\fcolorbox{black}{white}{\includegraphics[height=.75cm]{rc-logo}}
\end{textblock*}}
\setbeamertemplate{caption}[numbered]
%\setbeamerfont{caption}{size=\scriptsize}
\setbeamertemplate{itemize item}[triangle]
\setbeamertemplate{itemize subitem}[circle]
\setbeamertemplate{bibliography item}[triangle]
\setbeamertemplate{blocks}[rounded][shadow=true]


%%% COLORS %%%
\colorlet{Main}{black}

\setbeamercolor{author in head/foot}{fg=Main}
\setbeamercolor{title in head/foot}{fg=Main}
\setbeamercolor{date in head/foot}{fg=Main}
\setbeamercolor{page number in head/foot}{fg=Main}
\setbeamercolor{normal text}{fg=darkgray}
\setbeamercolor{structure}{fg=Main}
\setbeamercolor{itemize item}{fg=Main}
\setbeamercolor{itemize subitem}{fg=Main}
\setbeamercolor{block title}{bg=structure,fg=white}


%%% BIBLIOGRAPHY %%%
\usepackage[%
	backend=bibtex,
	citestyle=authoryear,
	maxcitenames=2,
	maxbibnames=99,
	firstinits=true,
	url=false,
	doi=false,
	isbn=false, 
	%sorting=none,
	]{biblatex}
%TODO \addbibresource{references.bib}
% use \footcite{jones00} or \footfullcite{jones00} to make footnote citations



%% TITLE PAGE INFO %%
\author{Anjana Vakil and Veronica Hanus} 
\institute{\includegraphics[height=4.5em]{rc-logo}\\Recurse Center}
\title[Intro to \LaTeX]{Introduction to \LaTeX}
\subtitle{A better way of typesetting documents} 
\date[11/19/15]{November 19, 2015}

%% DOCUMENT %%
\begin{document}
{
\setbeamertemplate{footline}{} 
\begin{frame}
  \titlepage
\end{frame}
}
\addtocounter{framenumber}{-1}

\begin{frame}
\frametitle{Outline}
\tableofcontents%[pausesections]
\end{frame}


\section{What is \LaTeX?}
\begin{frame}{What is \LaTeX?}
\begin{itemize}
\item A system/markup language for typesetting documents
\item Originally geared towards scientific/technical documents
\item Free software
\item Very customizable
\item Part of the greater \TeX~ family (\TeX, XeLaTeX, LuaLaTeX...)
\end{itemize}
\end{frame}

\section{Why use it?}
\begin{frame}{Use cases}
\begin{itemize}
\item Scholarly articles (vastly superior to Word etc.)
\item Theses, dissertations
\item Slides \& posters
\item $\dots$and more!
\end{itemize}
\end{frame}

\begin{frame}{Pros and cons}
Pros
\begin{itemize}
\item The de facto standard for academic articles
\item Many journals, conferences, etc. have style files
\item Very easy to include formulas, symbols, etc.
\item Version-control your documents!
\item Pre-existing packages for pretty much anything;\\If you really can't find one for feature X, write your own!
\item Huge, active community of users
\end{itemize}
\vfill
Cons:
\begin{itemize}
\item Learning curve; can be overwhelming at first
\item The perfect document set-up can take a while
\item You \textit{will} get weird errors (see slide \ref{resources})
\end{itemize}
\end{frame}


\section{\LaTeX~ basics}
\begin{frame}{\LaTeX~ basics}
\begin{block}{}
To see how a basic article works, take a look at \texttt{example/simple-example.tex}!
\end{block}
\end{frame}

\begin{frame}{Editing and compiling}

Simplest option:
\begin{enumerate}
\item Create and edit a \texttt{.tex} file in your favorite editor
\item At the command line run \texttt{pdflatex filename.tex}
\item View the beautiful \texttt{.pdf} file it generated and rejoice!
\end{enumerate}
\vfill
Other compilation/editing options:
\begin{itemize}
\item Sublime Text plugin (plugins also exist for other editors)
\item TeXmaker (cross-platform): \url{http://www.xm1math.net/texmaker/}
\item Other GUIs for Windows, Mac, Linux
\item Online collaborative editor: https://www.sharelatex.com/
\end{itemize}
\end{frame}

\section{Resources}
\label{resources}
\begin{frame}{Resources}
Documentation/guides:
\begin{itemize}
\item Wikibooks manual: \url{https://en.wikibooks.org/wiki/LaTeX}
\item ShareLaTeX documentation: \url{https://www.sharelatex.com/learn}
\end{itemize}
\vfill
Getting help:
\begin{itemize}
\item \TeX/\LaTeX~ Stack Exchange: \url{http://tex.stackexchange.com/}
\item Google is your friend!\\Someone else has definitely had problem X.
\end{itemize}
\end{frame}

\section{--Break--}
\begin{frame}{Break point!}
\begin{block}{}
\begin{center}
If you need to leave, now's a good time!

More details after the break.
\end{center}
\end{block}
\end{frame}

\section{Helpful commands}
\begin{frame}{Helpful commands}
\begin{block}{}
Take a look at \texttt{cheatsheet/cheatsheet.pdf}!
\end{block}
\end{frame}


\section{Using packages}
\begin{frame}{Using packages}
\begin{block}{}
\centering
In preamble: \texttt{\textbackslash usepackage\{packagename\}}
\end{block}
\begin{itemize}
\item Page layout: \texttt{\textbackslash usepackage[margin=1.5in]\{geometry\}}
\item Include image files: \texttt{graphicx}
\item Include web links: \texttt{hyperref} 
\item Display code: \texttt{verbatim} or \texttt{listings}
\item Control line spacing more easily: \texttt{setspace}
\item Customize headers/footers: \texttt{fancyhdr}
\item Customize figure/table captions: \texttt{caption}
\item More control over lists: \texttt{enumitem}
\end{itemize}
\end{frame}

\section{Math in \LaTeX}
\begin{frame}{Math in \LaTeX}
\begin{block}{Useful packages:}
\begin{itemize}
\item Advanced math \texttt{amsmath}, fixes \texttt{mathtools}
\item Options adjust all of single symbol (\texttt{\usepackage[<opt.>]{amsmath}})
\item Envirnment packages control layout of the whole document
\item GUI outputs code \url{https://www.codecogs.com/latex/eqneditor.php}
\end{itemize}
\end{block}

\vfill

Resources:
\begin{itemize}
\item Example-laden wiki to give you a jump-start \url{https://en.wikibooks.org/wiki/LaTeX/Mathematics}
\item Draw your symbol, find package needed \url{https://detexify.kirelabs.org}
\item GUI which outputs code \url{https://www.codecogs.com/latex/eqneditor.php}
\item Searchable community forum \url{https://www.codecogs.com/latex/eqneditor.php}
%\item Intense math guide (.pdf) \url{ftp://ftp.ams.org/pub/tex/doc/amsmath/short-math-guide.pdf}
\end{itemize}

\end{frame}

\section{Other useful features}
\begin{frame}{Figures and captions}

\begin{figure}
\includegraphics[scale=1.5]{rc-logo}
\caption{The RC logo}
\end{figure}
\end{frame}

\begin{frame}{Tables}

\begin{table}
	\caption{Recursers}
	\begin{tabular}{llr}
		\toprule
		Name 					& Pseudonym 				& Batch	\\
		\midrule
		Veronica Hanus 	& Flight Witch 				& Fall 2	\\
		Anjana Vakil 		& Spandex Governor 	& Fall 2	\\
		\bottomrule
	\end{tabular}
\end{table}

\begin{block}{Useful packages:}
\begin{itemize}
\item Controlling table widths: \texttt{tabularx}
\item Multi-row cells: \texttt{multirow}
\item Better formatting: \texttt{booktabs} 
\end{itemize}
\end{block}

\end{frame}

\begin{frame}{Slides with \texttt{beamer}}

\structure{Meta-example:} You're looking at it! Whoaaa!

\pause

\vfill

\begin{block}{Tip:}
The \texttt{beamer} class can be used for posters as well.
\end{block}

\pause

\vfill

Pitfalls:
\begin{itemize}[<+->]
\item Standard themes are uuuugly
\item Setting up a template you like can be a headache
\item Many academics who use \LaTeX~ for articles still use PowerPoint etc. for slides/posters
\end{itemize}
\end{frame}

\begin{frame}{Bibliographies}
\begin{block}{\texttt{.bib} files}
List all your sources in bibtex format.

Many citation systems (e.g. Google scholar, Mendeley) offer an ``export to bibtex'' option.
\end{block}
\begin{block}{\texttt{biblatex} Package}
Allows you to fully customize citation and bibliography styles.
\end{block}
\end{frame}



\end{document}