\documentclass[xcolor={dvipsnames}]{beamer}

%%% PACKAGES %%%
\usepackage{graphicx}
\usepackage{tabularx}
\usepackage{booktabs}
\usepackage{multirow}
%\usepackage[usenames,dvipsnames]{color}
\usepackage{textpos}
\usepackage{hyperref}




%%% FONT %%%
\renewcommand*{\familydefault}{phv}


%%% BEAMER THEME %%%
\usetheme{default}
\usecolortheme{whale}
\usecolortheme{orchid}
\setbeamertemplate{navigation symbols}{} 
\setbeamertemplate{footline}[frame number]
\addtobeamertemplate{frametitle}{}{%
\begin{textblock*}{100mm}(.99\textwidth,-.9cm)
\fboxsep=1pt%padding thickness
\fboxrule=1pt%border thickness
\fcolorbox{black}{white}{\includegraphics[height=.75cm]{rc-logo}}
\end{textblock*}}
\setbeamertemplate{caption}[numbered]
%\setbeamerfont{caption}{size=\scriptsize}
\setbeamertemplate{itemize item}[triangle]
\setbeamertemplate{itemize subitem}[circle]
\setbeamertemplate{bibliography item}[triangle]
\setbeamertemplate{blocks}[rounded][shadow=true]


%%% COLORS %%%
\colorlet{Main}{black}

\setbeamercolor{author in head/foot}{fg=Main}
\setbeamercolor{title in head/foot}{fg=Main}
\setbeamercolor{date in head/foot}{fg=Main}
\setbeamercolor{page number in head/foot}{fg=Main}
\setbeamercolor{normal text}{fg=darkgray}
\setbeamercolor{structure}{fg=Main}
\setbeamercolor{itemize item}{fg=Main}
\setbeamercolor{itemize subitem}{fg=Main}
\setbeamercolor{block title}{bg=structure,fg=white}


%%% BIBLIOGRAPHY %%%
\usepackage[%
	backend=bibtex,
	citestyle=authoryear,
	maxcitenames=2,
	maxbibnames=99,
	firstinits=true,
	url=false,
	doi=false,
	isbn=false, 
	%sorting=none,
	]{biblatex}
%TODO \addbibresource{references.bib}
% use \footcite{jones00} or \footfullcite{jones00} to make footnote citations



%% TITLE PAGE INFO %%
\author{Anjana Vakil and Veronica Hanus} 
\institute{\includegraphics[height=4.5em]{rc-logo}\\Recurse Center}
\title[Intro to \LaTeX]{Introduction to \LaTeX}
\subtitle{A better way of typesetting documents} 
\date[11/19/15]{November 19, 2015}

%% DOCUMENT %%
\begin{document}
{
\setbeamertemplate{footline}{} 
\begin{frame}
  \titlepage
\end{frame}
}
\addtocounter{framenumber}{-1}

\begin{frame}
\frametitle{Outline}
\tableofcontents%[pausesections]
\end{frame}


\section{What is \LaTeX?}
\begin{frame}{What is \LaTeX?}
\begin{itemize}
\item A system/markup language for typesetting documents
\item Originally geared towards scientific/technical documents
\item Free software
\item Very customizable
\item Part of the greater \TeX~ family (\TeX, XeLaTeX, LuaLaTeX...)
\end{itemize}
\end{frame}

\section{Why use it?}
\begin{frame}{Use cases}
%Uses (quick and flashy)
%beautiful typesetting
%review math, etc modules
%environments for report/book chapters, letters, figure labels, slides, posters, etc.
%templates for conferences etc.
%version control
\end{frame}

\section{Examples}
\begin{frame}{Examples}
%Examples of finished products
\end{frame}

\section{\LaTeX basics}
\begin{frame}{A basic document}
%Basic how-to
%article with basic info (author, title, sections, dummy text)
\end{frame}

\begin{frame}{Editing and compiling}
%command line compile
%
%Other compilation/editing options
%sublime plugin (also exist for other editors)
%TeXmaker and other GUIs
Online collaborative editor: https://www.sharelatex.com/
\end{frame}

\section{Limitations}
\begin{frame}{Limitations}
%Limitations
%learning curve
%not a editor! (?)
\end{frame}

\section{Resources}
\begin{frame}{Resources for learning more}
\begin{itemize}
\item Wikibooks manual: \url{https://en.wikibooks.org/wiki/LaTeX}
\item ShareLaTeX documentation: \url{https://www.sharelatex.com/learn}
\item \TeX/\LaTeX~ Stack Exchange: \url{http://tex.stackexchange.com/}
\end{itemize}
\end{frame}

\section{--Break--}
\begin{frame}{Break point!}
\begin{block}{}
\begin{center}
If you need to leave, now's a good time!

More details after the break.
\end{center}
\end{block}
\end{frame}

\section{Basic code examples}
\begin{frame}{Code Examples}
%arguments are named what you expect but take getting used to (large, Large, huge, etc.)
%Even more code! (examples of how much control you have)
%Hand out cheat sheet
\end{frame}


\section{Using packages}
\begin{frame}{Using packages}
\begin{block}{}
\centering
In preamble: \texttt{\textbackslash usepackage\{packagename\}}
\end{block}
\begin{itemize}
\item Page layout: \texttt{\textbackslash usepackage[margin=1.5in]\{geometry\}}
\item Include image files: \texttt{graphicx}
\item Include web links: \texttt{hyperref} 
\item Display code: \texttt{verbatim} or \texttt{listings}
\item Control line spacing more easily: \texttt{setspace}
\item Customize headers/footers: \texttt{fancyhdr}
\item Customize figure/table captions: \texttt{caption}
\item More control over lists: \texttt{enumitem}
\end{itemize}
\end{frame}

\section{Math in \LaTeX}
\begin{frame}{Math in \LaTeX}
%Math features
%use modes: LaTeX has text, display math, and inline math (super/subscripts are squeezed) modes
%what packages are best?
%most commands make sense (hell, maybe I should learn to math it up), will give a few examples
%for symbols: detexify.kirelabs.org
%math guide: ftp://ftp.ams.org/pub/tex/doc/amsmath/short-math-guide.pdf
\end{frame}

\section{Other useful features}
\begin{frame}{Figures and captions}

\begin{figure}
\includegraphics[scale=1.5]{rc-logo}
\caption{The RC logo}
\end{figure}
\end{frame}

\begin{frame}{Tables}

\begin{table}
	\caption{Recursers}
	\begin{tabular}{llr}
		\toprule
		Name 					& Pseudonym 				& Batch	\\
		\midrule
		Veronica Hanus 	& Flight Witch 				& Fall 2	\\
		Anjana Vakil 		& Spandex Governor 	& Fall 2	\\
		\bottomrule
	\end{tabular}
\end{table}

\begin{block}{Useful packages:}
\begin{itemize}
\item Controlling table widths: \texttt{tabularx}
\item Multi-row cells: \texttt{multirow}
\item Better formatting: \texttt{booktabs} 
\end{itemize}
\end{block}

\end{frame}

\begin{frame}{Slides with \texttt{beamer}}

\structure{Meta-example:} You're looking at it! Whoaaa!

\pause

\vfill

\begin{block}{Tip:}
The \texttt{beamer} class can be used for posters as well.
\end{block}

\pause

\vfill

Pitfalls:
\begin{itemize}[<+->]
\item Standard themes are uuuugly
\item Setting up a template you like can be a headache
\item Many academics who use \LaTeX~ for articles still use PowerPoint etc. for slides/posters
\end{itemize}
\end{frame}

\begin{frame}{Bibliographies}
\begin{block}{\texttt{.bib} files}
List all your sources in bibtex format.

Many citation systems (e.g. Google scholar, Mendeley) offer an ``export to bibtex'' option.
\end{block}
\begin{block}{\texttt{biblatex} Package}
Allows you to fully customize citation and bibliography styles.
\end{block}
\end{frame}



\end{document}